\documentclass[french,12pt,a4paper]{article}
\usepackage{fontspec}
\usepackage[french]{babel}
\usepackage{microtype}
\usepackage[utf8]{inputenc}
\usepackage{geometry}
\setmainfont[Ligatures=TeX,Scale=1.1]{EB Garamond}

\title{Audio et i-score}
\author{Jean-Michaël Celerier}

\begin{document}

\maketitle

\section{Introduction}
Ce document présente différentes possibilités 
pour l'écriture et l'utilisation de son dans i-score, 
et les environnements qui peuvent lui être adjoints.

\subsection{Problématique}
Les questions auxquelles ce document tente d'apporter une ou plusieurs réponses sont les suivantes : 
\begin{itemize}
    \item Quel doit être l'agencement des responsabilités entre i-score et les outils annexes pour avoir des possibilités d'écriture maximale.
    Cette question de partage de responsabilités se pose aussi au niveau des utilisateurs de ces outils : qui utillise i-score ? Qui utilise wWise ? Qui utilise Unity ? 
    \item Quelles sont les contraintes techniques qui peuvent s'appliquer lors de l'interopérabilité avec différents environnements, et comment les résoudre. 
    Par exemple, comment faire en sorte que l'environnement fonctionne sur mobile.
    \item Comment doit fonctionner la gestion des médias dans un projet complet.
    \item Comment doit fonctionner la répartition dans un cas ou l'on désire produire une \oe uvre distribuée; cela pose la question de la séparation du moteur d'édition et d'exécution dans le cas du son, ainsi que de leur communication.
    \item Quelles sont les problématiques d'écriture qui se pose lorsque l'on désire écrire des scènes sonores ? (Mute de certaines parties ? Collaboration à l'écriture ?)
    \item Un modèle de calcul a-t-il sa place dans l'environnement, et si oui, qui doit le fournir, et où ces calculs sont-ils écrits ?
    
\end{itemize}
- fonctionnement sur mobile / web ? iOS : pas d'IPC : https://developer.apple.com/library/ios/documentation/iPhone/Conceptual/iPhoneOSProgrammingGuide/Inter-AppCommunication/Inter-AppCommunication.html
Tout intégrer dans une appli ?
- Séparation écriture / exécution
- gestion des médias ?
- répartition ? (son sur téléphone)
- facilité de l'écriture : côté GAF ? côté BY ? côté artistes ?
- modèle de calcul ? veut-on écrire une cumulation d'effets dans i-score 
ou bien avoir tous les calculs faits à l'extérieur ?

\section{Types de moteurs audio}
- Audiographes
- Moteurs simples
- Logiciels (Max, etc.)
- Moteurs de jeu
- Moteurs déclaratifs
- Briques d'effets / de synthèse
- DSL

Autre possibilitié intéressante issue de Faust : utiliser LLVM pour recompiler 
à la volée du patch ?
\section{Gestion de la spatialisation}

\section{Briques en présence}
- Source de données spatiales fixées (carte, scénario, tableau, ...).
- i-score : scénarisation
- Source de données spatiales temps-réel (GPS, etc.).

\section{Possibilités d'implémentation}
Questions : 
* qui gère la sortie son ?
* qui gère la spatialisation en sortie (sur les hauts-parleurs)
* qui gère la spatialisation en entrée (des objets)
* qui applique des effets
* qui contrôle l'écoulement du temps scénaristique
* qui fait office de source sonore
* qui communique avec qui

- SuperCollider comme moteur audio ?
- Tout dans i-score ? 
- libaudiostream et la place de FaUST ?
- Grapholine ?
- Problème du contrôle si deux moteurs.

Cas 1. Séquenceur intégré à i-score
i-score devient live
Question principale : gestion de l'horloge
Possibilité : 
- Un processus s'enregistre aurpès du mixeur
- Quand le processus est démarré, le mixeur appelle pull() dessus
- Problématique : si le processus est déclenché via un évènement interactif, il est nécessaire de le prévoir (par exemple en commençant à appeler pull dès que l'on est dans la zone interactive, ou bien en utilisant des mutex / compteurs atomiques, ou bien en rajoutant de la latence, ou bie en mangeant le début.)
- Problématique : bufferisation si effet sonore met du temps à s'applique

Cas 2. Utilisation d'un séquenceur externe

Cas 3. Utilisation de i-score à haut niveau pour grands scénarios et à bas niveau pour objets sonores.
Entre les deux communications avec objets qui gèrent le son.

Cas 4. Utilisation de wWise 
Implémentation de i-score / grapholine / autre comme plug-in wWise ?

Pb. de la bufferistaion : temps-réel, latence ?

\section{Questions sémantiques}
Possibilté d'utilisation d'un outil comme OWL ?

\section{Utilisation pour les applications possibles}
- Tableaux, installs, etc.
- Morceaux de musiques interactifs que l'on peut distribuer par internet ?

\end{document}
